% 如果没有这一句命令,XeTeX会出错,原因参见
% http://bbs.ctex.org/viewthread.php?tid=60547
\DeclareRobustCommand\nobreakspace{\leavevmode\nobreak\ }

%------------------------------标题名称中文化-----------------------------%
\renewcommand{\chaptername}{第\CJKnumber{\thechapter}章}

%\renewcommand\abstractname{\hei 摘\ 要}
\renewcommand\refname{\hei 参考文献}
\renewcommand\figurename{\hei 图}
\renewcommand\tablename{\hei 表}

\addtolength{\headsep}{-0.1cm}          %页眉位置
\addtolength{\footskip}{-0.1cm}         %页脚位置
\oddsidemargin 8pc \evensidemargin 8pc
\addtolength{\oddsidemargin}{-1in}% compensate for LaTeX's 1in offset
\addtolength{\evensidemargin}{-1in}% compensate for LaTeX's 1in offset

\setlength{\abovedisplayskip}{6pt plus1pt minus1pt}     %公式前的距离
\setlength{\belowdisplayskip}{6pt plus1pt minus1pt}     %公式后面的距离
\setlength{\arraycolsep}{4pt}   %在一个array中列之间的空白长度, 因为原来的太宽了

\renewcommand{\textfraction}{0.15}
\renewcommand{\topfraction}{0.85}
\renewcommand{\bottomfraction}{0.65}
\renewcommand{\floatpagefraction}{0.60}

\newlength \CJKtwospaces


\setcounter{tocdepth}{3} \setcounter{secnumdepth}{3}


%厦大标准:
%章的标题:小3号加重黑体;
\titleformat{\chapter}[hang]
    {\normalfont\xiaosan\filcenter\bf\hei}
    {\xiaosan{\chaptertitlename}}
    {20pt}{\xiaosan}
%\titlespacing{\chapter}{0pt}{-3ex  plus .1ex minus .2ex}{2.5ex plus .1ex minus .2ex}
\titlespacing{\chapter}{0pt}{-3ex  plus .1ex minus .2ex}{0.25em}

%节的标题:4号加重黑体;
\titleformat{\section}[hang]{\bf \hei \sihao}
    {\sihao \thesection}{0.5em}{}{}
%\titlespacing{\section}{0pt}{1.5ex plus .1ex minus .2ex}{\wordsep}
\titlespacing{\section}{0pt}{-0.2em}{0.8em}

%目及子目以下的标题:小4号加重黑体;
\titleformat{\subsection}[hang]{\bf \hei \xiaosi}
    {\xiaosi \thesubsection}{0.5em}{}{}
%    {\banxiaosi \thesubsection}{0pt}{}{}
%\titlespacing{\subsection}{0pt}{1.5ex plus .1ex minus .2ex}{\wordsep}
\titlespacing{\subsection}{0pt}{-0.25em}{1em}

\titleformat{\subsubsection}[hang]{\bf \hei \xiaosi}
    {\thesubsubsection }{0.5em}{}{}
%\titlespacing{\subsubsection}{0pt}{1.2ex plus .1ex minus .2ex}{\wordsep}
\titlespacing{\subsubsection}{0pt}{0.25em}{0pt}

%去掉中间对齐的sectionformat,这样就把节的标题左对齐了。
%\renewcommand \sectionformat{}

%% 按清华标准, 缩小目录中各级标题之间的缩进
%% 按厦大标准, 目录中:章的标题的字号:四号黑体加重;节的标题的字号:小四黑体加重;目的标题的字号:小四宋体。
\renewcommand\contentsname{目\quad 录}

\dottedcontents{chapter}[2.86em]{\bf \hei \sihao \vspace{0.5em}}{3.4em}{5pt}
\dottedcontents{section}[1.16cm]{\bf \hei \xiaosi}{1.8em}{5pt}
\dottedcontents{subsection}[2.00cm]{\song \xiaosi}{2.7em}{5pt}
\dottedcontents{subsubsection}[2.86cm]{\song \xiaosi}{3.4em}{5pt}


%%去掉章节标题中的数字
%\renewcommand{\sectionmark}[1]{\markright{\thesection \ #1}}
%\renewcommand{\chaptermark}[1]{\markboth{\chaptername \ #1}{}}
%%% Clear Header %%%%%%%%%%%%%%%%%%%%%%%%%%%%%%%%%%%%%%%%%%%%%%%%%%%
% Clear Header Style on the Last Empty Odd pages
\makeatletter
\def\cleardoublepage{\clearpage\if@twoside \ifodd\c@page\else%
    \hbox{}%
    \thispagestyle{empty}%              % Empty header styles
    \newpage%
    \if@twocolumn\hbox{}\newpage\fi\fi\fi}

%%%%%%%%%%%%%%%%%%%%%%%%%%%%%%设置符号型脚注%%%%%%%%%%%%%%%%
%\renewcommand{\thefootnote}{\textcircled{\arabic{footnote}}}
\renewcommand{\thefootnote}{\textcircled{\normalsize\tiny\arabic{footnote}} }% 文章注释改为加圈的上标
%%%%%%%%%%%%%%%%%%%%%%%%%%%%%%%%%%%%%%%%%%%%%%%%%%%%%%%%
% 设置行距和段落间垂直距离
%%%%%%%%%%%%%%%%%%%%%%%%%%%%%%%%%%%%%%%%%%%%%%%%%%%%%%%%

% 段落之间的竖直距离
\setlength{\parskip}{3pt plus1pt minus1pt}

% 定义行距
\renewcommand{\baselinestretch}{1.2}

%%%%%%%%%%%%%%%%%%%%%%%%%%%%%%%%%%%%%%%%%%%%%%%%%%%%%%%%
% 调整列表环境的垂直间距
%%%%%%%%%%%%%%%%%%%%%%%%%%%%%%%%%%%%%%%%%%%%%%%%%%%%%%%%
\let\orig@Itemize =\itemize
\let\orig@Enumerate =\enumerate
\let\orig@Description =\description

\def\Myspacing{\itemsep=0ex \topsep=0ex \partopsep=0ex \parskip=0ex \parsep=0ex}
%%\def\Myspacing{\itemsep=1.0ex \topsep=0ex \partopsep=0pt \parskip=0pt \parsep=0.5ex}

\def\newitemsep{
\renewenvironment{itemize}{\orig@Itemize\Myspacing}{\endlist}
\renewenvironment{enumerate}{\orig@Enumerate\Myspacing}{\endlist}
\renewenvironment{description}{\orig@Description\Myspacing}{\endlist}
}

\def\olditemsep{
\renewenvironment{itemize}{\orig@Itemize}{\endlist}
\renewenvironment{enumerate}{\orig@Enumerate}{\endlist}
\renewenvironment{description}{\orig@Description}{\endlist}
}

\newitemsep

%%%%%%%%%%%%%%%%%%%%%%%%%%%%%%%%%%%%%%%%%%%%%%%%%%%%%%%
% 修改引用的格式,
%%%%%%%%%%%%%%%%%%%%%%%%%%%%%%%%%%%%%%%%%%%%%%%%%%%%%%%

%第一行在引用处数字两边加方框
%第二行去除参考文献里数字两边的方框
%\makeatletter
%\def\@cite#1{\mbox{$\m@th^{\hbox{\@ove@rcfont[#1]}}$}}
%\renewcommand\@biblabel[1]{#1}
%\makeatother

% 修改 \citep 命令使显示的引用为上标形式
\renewcommand{\citep}[1]{$^{\mbox{\scriptsize \cite{#1}}}$}

%%%%%%%%%%%%%%%%%%%%%%%%%%%%%%%%%%%%%%%%%%%%%%%%%%%%%%%%%%%
%
% 定制浮动图形和表格标题样式
%
%%%%%%%%%%%%%%%%%%%%%%%%%%%%%%%%%%%%%%%%%%%%%%%%%%%%%%%%%%%
% 按清华标准, 去掉图表号后面的:
\renewcommand{\arraystretch}{1}

%\renewcommand{\captionfont}{\CJKfamily{song}\rmfamily}
%\renewcommand{\captionlabelfont}{\CJKfamily{song}\rmfamily}
%厦大标准:编号,表名(小4号宋体加重)
\renewcommand{\captionfont}{\xiaosi \song \bf}
\renewcommand{\captionlabelfont}{\xiaosi \song \bf}

%%%%%%%%%%%%%%%%%%%%%%%%%%%%%%%%%%%%%%%%%%%%%%%%%%%%%%%%%%%%%%%%%%%%%%
% 自定义项目列表标签及格式 \begin{denselist} 列表项 \end{denselist}
%%%%%%%%%%%%%%%%%%%%%%%%%%%%%%%%%%%%%%%%%%%%%%%%%%%%%%%%%%%%%%%%%%%%%%
\newcounter{newlist} %自定义新计数器
\newenvironment{denselist}[1][temp]{%%%%%定义新环境:可改变的列表题目
\begin{list}{\textbf{#1} (\arabic{newlist})} %%标签格式
    {
    \usecounter{newlist}
     \setlength{\labelwidth}{22pt} %标签盒子宽度
     \setlength{\labelsep}{0cm} %标签与列表文本距离
     \setlength{\leftmargin}{0cm} %左右边界
     \setlength{\rightmargin}{0cm}
     \setlength{\parsep}{0ex} %段落间距
     \setlength{\itemsep}{0ex} %标签间距
     \setlength{\itemindent}{40pt} %标签缩进量
     \setlength{\listparindent}{40pt} %段落缩进量
    }}
{\end{list}}%%%%%

\def\defaultfont{\renewcommand{\baselinestretch}{1.5}
\fontsize{12pt}{13pt}\selectfont}


%%%%%%%%%%%%%%%%%%%%%%%%%%%%%%%%%%%%%%%%%%%%%%%%%%%%%%%%%%%%%%%%%%%%%%
% 封面、摘要、版权、致谢格式定义
%%%%%%%%%%%%%%%%%%%%%%%%%%%%%%%%%%%%%%%%%%%%%%%%%%%%%%%%%%%%%%%%%%%%%%
\def\ctitle#1{\def\@ctitle{#1}}\def\@ctitle{}
\def\cdegree#1{\def\@cdegree{#1}}\def\@cdegree{}
\def\caffil#1{\def\@caffil{#1}}\def\@caffil{}
\def\csubject#1{\def\@csubject{#1}}\def\@csubject{}
\def\cauthor#1{\def\@cauthor{#1}}\def\@cauthor{}
\def\cnumber#1{\def\@cnumber{#1}}\def\@cnumber{}
\def\csubsub#1{\def\@csubsub{#1}}\def\@csubsub{}
\def\csupervisor#1{\def\@csupervisor{#1}}\def\@csupervisor{}
\def\cdate#1{\def\@cdate{#1}}\def\@cdate{}
\def\ddate#1{\def\@ddate{#1}}\def\@ddate{}
\long\def\cabstract#1{\long\def\@cabstract{#1}}\long\def\@cabstract{}
\def\ckeywords#1{\def\@ckeywords{#1}}\def\@ckeywords{}

\def\etitle#1{\def\@etitle{#1}}\def\@etitle{}
\long\def\eabstract#1{\long\def\@eabstract{#1}}\long\def\@eabstract{}
\def\ekeywords#1{\def\@ekeywords{#1}}\def\@ekeywords{}


\def\makecover{
    \begin{titlepage}
    %中文封面%%%%%%%%%%%%%%%%%%%%%%%%%%%%%%%%%%%%%%%%%%%%%%%%%%%
    \newpage
    \thispagestyle{empty}

    \begin{minipage}[b][0cm][b]{9cm}
    \begin{flushleft}
    \song \xiaosi \bf {学校编码:10384}\\
    \song \xiaosi \bf {学号:\@cnumber}
    \end{flushleft}

    \end{minipage} \hfill

    \parbox[r][0cm][r]{13.6cm}{
    \begin{flushright}
    \song \xiaosi \bf {分类号\underline{~~~~~~~~~~~~~~}密级\underline{~~~~~~~~~~~~}}\\
    \song \xiaosi \bf {UDC\underline{~~~~~~~~~~~~~~}}
    \end{flushright}}

    \begin{center}

    \begin{figure}[htb!]
        \centering
        \includegraphics[width=6.2cm,bb=0 0 200 58]{figures/xmulogo.jpg}
    \end{figure}

    \parbox[t][1cm][c]{\textwidth}{\xiaoer
    \begin{center} {
    \song \bf \@cdegree ~~~~学~~~~位~~~~论~~~~文}\end{center} }

    \parbox[t][5.5cm][c]{\textwidth}{
    \begin{center}{\erhao \hei \bf \@ctitle }\end{center}
    \begin{center}{\sanhao \bf \@etitle }\end{center}}

    \parbox[t][1cm][c]{\textwidth}{\xiaoer\kai
    \begin{center} {\@cauthor}
    \end{center} }

    \parbox[t][5.5cm][c]{\textwidth}{ {\sihao\kai
    \begin{center}
    \begin{tabular}{cc@{\extracolsep{1em}}l}
    ~ & 指导教师姓名:& \@csupervisor \\
    ~ & 专~~~业~~~名~~~称:& \@csubject \\
    ~ & 论文提交日期:& \@ddate\\
    ~ & 论文答辩时间:& \@ddate\\
    ~ & 学位授予日期:& \@ddate\\
    \end{tabular}
    \end{center} } }

    \parbox[t][2cm][r]{\textwidth}{
    \begin{flushright}
    \song \sihao  {答辩委员会主席:\underline{~~~~~~~~~~~~~~~~~~~~}~~~~~~~~~~~~}\\
    \song \sihao  {评~~~~~~~~阅~~~~~~~~人:\underline{~~~~~~~~~~~~~~~~~~~~}~~~~~~~~~~~~}\\
    \end{flushright}}

    \parbox[b][2cm][b]{\textwidth}{
    \begin{center} {\sihao \song \@cdate~年~~~~~~~~月} \end{center} }
    \end{center}

    % 封二 空白页
    \cleardoublepage
    \end{titlepage}
}

\def\makeabstract{
    \addcontentsline{toc}{chapter}{\hei 摘要}
    \defaultfont

    %\chapter*{摘要}
    \pagestyle{plain}
    %\markboth{中~文~摘~要}{中~文~摘~要}
    \begin{center} {\sanhao \textbf{摘要}}\\
    \end{center}
    \setcounter{page}{1}

    \@cabstract

    \vspace{1em}
    \noindent {\hei 关键词:}\@ckeywords

    \cleardoublepage

    %英文摘要%%%%%%%%%%%%%%%%%%%%%%%%%%%%%%%%%%%%%%%%%%%%%%%%%%%%%
    \defaultfont
    \newpage
    \addcontentsline{toe}{chapter}{Abstract}
    \pagestyle{plain}
    %\chapter*{}
    %\markboth{英~文~摘~要}{英~文~摘~要}
    \begin{center} {\sanhao \textbf{Abstract}}\\
    \end{center}
    \renewcommand{\baselinestretch}{1.5}
    \fontsize{12pt}{12pt}\selectfont
    \@eabstract

    \vspace{1em}
    \noindent {\textbf{Key Words:}} \quad \@ekeywords
    \cleardoublepage
}

\long\def\statement{
    \newpage
    \thispagestyle{empty}
    \defaultfont

    \begin{center}
    \parbox[t][2cm][t]{\textwidth}{ {\xiaoer \song \centerline {\textbf{厦门大学学位论文原创性声明}}}}
    \parbox[t][9cm][c]{1.01\textwidth}{ \song \sihao
    \hspace{2em}本人呈交的学位论文是本人在导师指导下,独立完成的研究成果。
    本人在论文写作中参考其他个人或集体已经发表的研究成果,
    均在文中以适当方式明确标明,并符合法律规范和《厦门大学研究生学术活动规范(试行)》。

    \hspace{2em}另外,该学位论文为(\hspace{11em})课题(组)的研究成果,
    获得(\hspace{8em})课题(组)经费或实验室的资助,在
    (\hspace{7em})实验室完成。(请在以上括号内填写课题或课题组负责人或实验室名称,未有此项声明内容的,可以不作特别声明。)
}
    \end{center}

    \parbox[t][4cm][b]{\textwidth}{
    \begin{flushright}
    \song \xiaosi 声明人(签名):~~~~~~~~~~~~~~~~~~~~~~~~~~~~~~~~ \\
    \song \xiaosi 年~~~~~~~~月~~~~~~~~日~~~~~~~~~~~~~~~~
    \end{flushright}}
    \cleardoublepage
}

\def\copyright{
    \newpage
    \thispagestyle{empty}
    \defaultfont
    \begin{center}
    \parbox[t][2cm][t]{\textwidth}{ {\xiaoer \song \centerline {\textbf{厦门大学学位论文著作权使用声明}} } }
    \parbox[t][15cm][c]{1.02\textwidth}{ \song \sihao
    \hspace{2em}本人同意厦门大学根据《中华人民共和国学位条例暂行实施办法》等规定保留和使用此学位论文,
    并向主管部门或其指定机构送交学位论文(包括纸质版和电子版),
    允许学位论文进入厦门大学图书馆及其数据库被查阅、借阅。
    本人同意厦门大学将学位论文加入全国博士、硕士学位论文共建单位数据库进行检索,
    将学位论文的标题和摘要汇编出版,采用影印、缩印或者其它方式合理复制学位论文。\\
      本学位论文属于:\\
      (\hspace{3em})1.经厦门大学保密委员会审查核定的保密学位论文,
                 于  年  月  日解密,解密后适用上述授权。\\
      (\hspace{3em})2.不保密,适用上述授权。\\
      (请在以上相应括号内打“√”或填上相应内容。
        保密学位论文应是已经厦门大学保密委员会审定过的学位论文,
        未经厦门大学保密委员会审定的学位论文均为公开学位论文。此声明栏不填写的,默认为公开学位论文,均适用上述授权。)
    }
    \end{center}

    \parbox[t][4cm][c]{\textwidth}{\song \xiaosan
    \begin{flushright}
    \song \xiaosi 声明人(签名):~~~~~~~~~~~~~~~~~~~~~~~~~~~~~~~~ \\
    \song \xiaosi 年~~~~~~~~月~~~~~~~~日~~~~~~~~~~~~~~~~
    \end{flushright}
    }

    \cleardoublepage
}

\makeatletter
\def\hlinewd#1{%
  \noalign{\ifnum0=`}\fi\hrule \@height #1 \futurelet
   \reserved@a\@xhline}
\makeatother
%定义索引生成
\def\generateindex{
    \addcontentsline{toc}{chapter}{\indexname}

    \printindex
    \cleardoublepage
}

\raggedbottom

%\oddsidemargin 0.9em

\makeatletter
\newcommand\engcontentsname{Contents}
\newcommand\tableofengcontents{%
    \if@twocolumn
      \@restonecoltrue\onecolumn
    \else
      \@restonecolfalse
    \fi
    \chapter*{\engcontentsname
        \@mkboth{%
           \MakeUppercase\engcontentsname}{\MakeUppercase\engcontentsname}}%
    \@starttoc{toe}% !!!!Define a new contents!!!!
    \if@restonecol\twocolumn\fi
    }
\newcommand\addengcontents[2]{%
    \addcontentsline{toe}{#1}{\protect\numberline{\csname the#1\endcsname}#2}}
\makeatother

\newcommand\echapter[1]{\addengcontents{chapter}{#1}}
\newcommand\esection[1]{\addengcontents{section}{#1}}
\newcommand\esubsection[1]{\addengcontents{subsection}{#1}}
